\documentclass[a4paper,12 pt]{article}
\usepackage{color}
\usepackage[bahasa]{babel}
\usepackage{graphicx}
\graphicspath{ {./images/} }
\usepackage{blindtext}

%membuat url bisa diklik
\usepackage{hyperref}
\hypersetup{
    colorlinks,
    citecolor=black,
    filecolor=black,
    linkcolor=blue,
    urlcolor=blue
}

\title{\textbf{Tugas Membuat Dokumen }\linebreak 
\textbf{Dengan \LaTeX}\linebreak}
\date{}


\begin{document}

\maketitle
\thispagestyle{empty}
\begin{center}
\includegraphics[width=4cm,height=5cm]{logo}
\end{center}


\vspace{0.5 cm}
\begin{center}
\begin{tabular}{ll}
Nama & : Nama Mahasiswa disini \\
NIM & : xxxxxx\\
\end{tabular}
\newline
\newline
\newline
Untuk Memenuhi Tugas Mata Kuliah \\
Dosen Pengampu: Apri Junaidi, S. Kom., M. Kom., MCS \linebreak
\newline
\newline
\textbf {PROGRAM STUDI SAINS DATA} \\
\textbf {FAKULTAS INFORMATIKA} \\
\textbf {INSTITUT TEKNOLOGI TELKOM PURWOKERTO}
\linebreak
\textbf {2021} \linebreak
\end{center}

\pagebreak

\tableofcontents
\pagebreak

\listoffigures
\pagebreak

\section{Menampilkan Gambar di \LaTeX}
\blindtext\\
\linebreak
Ini adalah contoh untuk  menyisipkan gambar dalam dokumen seperti terlihat pada Gambar \ref{fig:logo}.


\begin{figure}[htp]
	\centering
	\includegraphics[width=4cm,height=5cm]{logo}
	\caption{Logo Institut Teknologi Telkom Purwokerto}
	\label{fig:logo}
\end{figure}

\section{Menampilkan script Program di \LaTeX}
\blindtext\\
\linebreak
Berikut adalah script program python untuk menghitung perkalian :

\begin{verbatim}
ulang = int(input ("Jumlah Looping : "))
kali = int(input ("Perkalian berapa :")) 

for i in range (1, ulang + 1) :
  print(i, "x", kali, "=", i * kali )
print ("Looping sudah dilakukan")
\end{verbatim}


\end{document}